% 字号设置
\newcommand{\chuhao}{\fontsize{42pt}{\baselineskip}\selectfont}     % 字号设置
\newcommand{\xiaochuhao}{\fontsize{36pt}{\baselineskip}\selectfont} % 字号设置
\newcommand{\yichu}{\fontsize{32pt}{\baselineskip}\selectfont}      % 字号设置
\newcommand{\yihao}{\fontsize{28pt}{\baselineskip}\selectfont}      % 字号设置
\newcommand{\erhao}{\fontsize{21pt}{\baselineskip}\selectfont}      % 字号设置
\newcommand{\xiaoerhao}{\fontsize{18pt}{\baselineskip}\selectfont}  % 字号设置
\newcommand{\sanhao}{\fontsize{15.75pt}{\baselineskip}\selectfont}  % 字号设置
\newcommand{\sihao}{\fontsize{14pt}{\baselineskip}\selectfont}      % 字号设置
\newcommand{\xiaosihao}{\fontsize{12pt}{\baselineskip}\selectfont}  % 字号设置
\newcommand{\wuhao}{\fontsize{10.5pt}{\baselineskip}\selectfont}    % 字号设置
\newcommand{\xiaowuhao}{\fontsize{9pt}{\baselineskip}\selectfont}   % 字号设置
\newcommand{\liuhao}{\fontsize{7.875pt}{\baselineskip}\selectfont}  % 字号设置
\newcommand{\qihao}{\fontsize{5.25pt}{\baselineskip}\selectfont}    % 字号设置

% 下划线
\newcommand{\underlineFixlen}[2][3.5cm]{\underline{\makebox[#1][c]{#2}}}

\newcommand{\keywords}[1][XXX;XXX;XXX;XXX;XXX]{#1}
\newcommand{\keywordsEn}[1][×××; ×××; ×××; ×××; ×××]{#1}

% 中文摘要
\renewenvironment{abstract}{
\thispagestyle{empty} % 去掉页码
{
\begin{center}
\Large \songti \bfseries 摘\hspace{1em}要\vspace{1.1cm}
\end{center}
}
\setlength{\parindent}{2em}
\setlength{\parskip}{0em}
\setlength{\baselineskip}{22pt} % (宋体,小四;固定行距22磅,段前、段后均为0行间距。段落首行缩进2字符。)
\songti
}{
\setlength{\parindent}{0em}
\setlength{\parskip}{1em}
{\par \songti \bfseries{关键词:}}
\keywords
\clearpage
}

% 英文摘要
\newenvironment{abstractEn}{
\thispagestyle{empty} % 去掉页码
{
\begin{center}
\Large \bfseries ABSTRACT\vspace{1.5cm}
\end{center}
}
\setlength{\parindent}{1em}
\setlength{\parskip}{0em}
\setlength{\baselineskip}{22pt} % 22磅行距,首行缩进1字符,段前、段后均为0行间距
}{
\setlength{\parindent}{0em}
\setlength{\parskip}{1em}
{\par \bfseries{KEYWORDS:}}
\keywordsEn
\clearpage
}

% 目录
\renewcommand\contentsname{
{
\begin{center}
\songti \Large \bfseries 目\hspace{1em}录 % (宋体,小二号,加粗;居中,单倍行距,段前0.5行、段后1.5行间距)
\end{center}
}
}

\renewcommand\refname{\heiti \sanhao \bfseries 参考文献}

\renewcommand{\cftdotsep}{1.5} % 线的密度
\renewcommand{\cftsecdotsep}{1.5} % section引线
\renewcommand{\cftsecleader}{\cftdotfill{\cftsecdotsep}}
\renewcommand{\cftsecpagefont}{}
